% 默认页面大小 4:3
\documentclass[10pt]{ctexbeamer}
% 页面大小 16:10
% \documentclass[10pt, aspectratio=1610]{ctexbeamer}
% 页面大小 16:9
% \documentclass[10pt, aspectratio=169]{ctexbeamer}
% 页面大小 14:9
% \documentclass[10pt, aspectratio=149]{ctexbeamer}
% 页面大小 1.41:1
% \documentclass[10pt, aspectratio=141]{ctexbeamer}
% 页面大小 5:4
% \documentclass[10pt, aspectratio=54]{ctexbeamer}
% 页面大小 3:2
% \documentclass[10pt, aspectratio=32]{ctexbeamer}

\usetheme[logo=litluglogo]{litlug}


% 引入参考文献列表的 .bib 文件, 使用 GB/T 7714-2015 的文献著录规则.
\usepackage[backend=biber, style=gb7714-2015]{biblatex}
\addbibresource{ref.bib}

\title[LITLUG Beamer (\LaTeX{})]{Linux用户组 Beamer (\LaTeX{})}
\subtitle[]{GNU/Linux User Group}
\author[AjiaErin]{\href{mailto:icpove@gmail.com}{AjiaErin}}
\institute{洛阳理工学院电气工程与自动化学院}

\subject{展示主题}
\keywords{展示, 关键词}

\begin{document}

\begin{frame}[plain]
  \maketitle
\end{frame}

\begin{frame}[t]
  \frametitle{目录}
  \tableofcontents
\end{frame}

\section[第 1 章缩写标题]{第 1 章标题}\label{sec:1}
\subsection[第 1 节缩写标题]{第 1 节标题}\label{subsec:1-1}

\begin{frame}[t]
  \frametitle{幻灯片标题}
  \framesubtitle{幻灯片副标题}
  平凡格式\quad\structure{浅灰格式}\quad\alert{强调格式}
  \begin{itemize}
    \item 第一级文本内容
    \item 若该行文本内容十分长长长长长长长长长则会被强制换行
      这里也可以包含\alert{需要强调的文本}
      \begin{itemize}
        \item 第二级文本内容
          \begin{itemize}
            \item 第三级文本内容
          \end{itemize}
          \alert{\item 第二级强调的文本内容}
      \end{itemize}
  \end{itemize}
  \pause% 动态展示以下内容
  \begin{enumerate}
    \item 带序号的文本内容
      \begin{enumerate}
        \item 第二级文本内容且包含数学公式
          \[\int^{\infty}_{-\infty}e^{-x^2}dx = \sqrt{\pi}\]
      \end{enumerate}
  \end{enumerate}
\end{frame}

\subsection[第 2 节缩写标题]{第 2 节标题}\label{subsec:1-2}

\begin{frame}[t]
  \frametitle{文本区块}
  将文本放入区块内
  \begin{block}{普通区块}
    黑色
  \end{block}
  \begin{exampleblock}{示例区块}
    \red{红色}
  \end{exampleblock}
  \begin{alertblock}{强调区块}
    \green{绿色}
  \end{alertblock}
  脚注\footnote{\url{https://github.com/lit-lug/Beamer}}
\end{frame}


\subsection[第 3 节缩写标题(虽然是缩写也可以很长长长长长长)]{第 3 节标题}\label{subsec:1-3}

\begin{frame}[t]
  \frametitle{表格}
  \begin{table}
    \begin{tabular}{lcl}\toprule
      姓名   & 出生年份 & 母校 (本科)  \\ \midrule
      陶哲轩 & 1975     & 弗林德斯大学 \\
      张益唐 & 1955     & 北京大学     \\
      丘成桐 & 1949     & 香港中文大学 \\ \bottomrule
    \end{tabular}
    \caption{二十一世纪的数学家}
  \end{table}
\end{frame}

\begin{frame}[t]
  \frametitle{引用}
  \begin{theorem}[\citeauthor{graham1989concrete}\citeyear{graham1989concrete}]\label{thm1}
    \[\textbf{具体 (\red{Concrete})} = \textbf{连续 (\red{Con}tinuous)} + \textbf{离散 (Dis\red{crete})}\]
  \end{theorem}
  \vfill
  \begin{proof}
    证明过程.
  \end{proof}
  引用定理~\ref{thm1}.
  \blfootnote{\printbibliography[heading=none,keyword={concrete}]}
\end{frame}

\section[第 2 章缩写标题也可以很长长长]{第二章标题}\label{sec:2}
\subsection*{第 4 节缩写标题 (不在目录显示)}\label{sec:2-1}
\makeatletter
\begin{frame}[t]
  \frametitle{自定义字体大小}
  \begin{center}
    \begin{tabular}{ll}
      \Huge  $\backslash$Huge                & \Huge \structure{24.88 pt}     \\
      \huge  $\backslash$huge                & \huge \structure{20.74 pt}     \\
      \LARGE $\backslash$LARGE               & \LARGE \structure{17.28 pt}    \\
      \Large $\backslash$Large               & \Large \structure{14.4 pt}     \\
      \large $\backslash$large               & \large \structure{12 pt}       \\
      \normalsize $\backslash$normalsize     & \normalsize \structure{10 pt}  \\
      \small $\backslash$small               & \small \structure{9 pt}        \\
      \footnotesize $\backslash$footnotesize & \footnotesize \structure{8 pt} \\
      \scriptsize $\backslash$scriptsize     & \scriptsize \structure{7 pt}   \\
      \tiny $\backslash$tiny                 & \tiny \structure{5 pt}
    \end{tabular}
  \end{center}
\end{frame}
\makeatother

\begin{frame}[noframenumbering, allowframebreaks, t]
  \frametitle{参考文献}
  \nocite{*}% 打印未引用,但已列入 .bib 文件内的文献
  \printbibliography%
\end{frame}

\begin{frame}[plain]
  \vfill
  \centerline{\Huge 感谢聆听}
  \vfill
\end{frame}

\end{document}
